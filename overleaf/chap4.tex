\chapter{Future work \& Conclusions}
\pagestyle{fancy}
% \lhead{Conclusions}
\label{futurework}

This project tackled the shortcomings of a foreseeable problem, dealing with a complex environment that requires interdisciplinary research. The type of data collected for this application requires robust knowledge, curiosity, and openness to failure as now and then, something will break and data will be lost. Similar problems that were recently exposed to new technological innovation struggle at the very same levels - the lack of existing datasets, powerful enough infrastructure, software tools to properly model the problem. This application is the first step towards a proper infrastructure that, over time, will collect enough data to improve its underlying models and represent a foundation for other software engineering challenges.

% In this chapter we will discuss about certain aspects that can be improved, as this project is worth continuing, for example, at the Master's Degree. An important aspect that can be improved is the user interaction with the application. 

Due to the limited resources, one weather station and data from only one use in the proximity of the weather station, features such as register/login were not bringing additional value to the project. Even so, in order to continue the development, we have to improve the application by creating a complete flow of actions through which a user can obtain autonomy over the Strava account or the weather station it wants to have.

The prototype is in a proof-of-concept stage, requiring a redesign in terms of exterior parts and also hardware components. It should be autonomous, or require very few configuration procedures. Only of these improvements can be considered stand-alone projects as they require focus on very specific areas. Throughout this project we made extra steps, adding support and making this system a fertile environment for additional features. 

\clearpage
Finally, ...

\lhead{}